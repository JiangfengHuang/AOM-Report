
%------------------------------------------------------------------------------
\subsubsection{Fluorescence Microscopy}
\label{sec:FlourescnecMicroscopy}

next section, all from \cite{wide_AOM_FM_spehrical_correction} uses a 
standard widefield flourescene microscope but use AOM to correct for 
spherical aberration due to depth -> no specimen induce correction
uses deconvolution to get out of foucs photons corrected

In this paper, we concentrate on the depth dependent aberration which can 
quickly become serious. Imaging 20 mum a live sample (index of refraction 1.36
) with an oil immersion lens causes the peak intensity of the point spread 
function (PSF) to drop 3-fold and the width of the PSF in the axial direction 
to increase by 2-folds. \cite{wide_AOM_FM_spehrical_correction} 

Because wide-field microscopy captures as efficiently as possible every 
emitted photon ultimately minimizing the sample excitation dose, it is well 
suited to in vivo imaging in samples where scattering is not too large. 
Although the out-of-focus photons are in the wrong place, they can be 
effectively re-assigned to the location of emission by constrained 
deconvolution algorithms \cite{wide_deconvolution}

The problem of depth aberrations can be solved by matching the sample index 
and the index of the immersion medium, but this is frequently not feasible or 
desirable. For example, the index of fixed cells can be matched to that of 
the immersion oil, but this option is not available for live imaging.

An important drawback to most schemes that have been proposed so far is that 
they require several images to be taken to optimize the aberration 
correction. This presents a serious problem for live imaging in biology 
because the fluorescence intensities can be weak and susceptible to rapid 
bleaching.

The approach we follow is to correct the depth aberrations withanopen-loop 
predictive algorithm similar totheapproach taken by Potsaid et al. in 
correcting off-axis aberrations. This is possible because the depth 
aberration can be calculated for a given depth into the sample. The depth 
aberration is the result of depth-dependent path length differences.

Correcting depth aberrations with a DM improves both the peak intensities and 
the deconvolution of images taken below the cover slip by removing the depth 
aberration. This allows the use of fast space-invariant deconvolution 
algorithms instead of depth-dependent algorithms. This is significant because 
it improves both the signal-to-noise ratio and the resolution in biological 
imaging where photons are in short supply. Unfortunately, the performance 
does not yet achieve what is theoretically possible.

\begin{figure}[htb]
	\centering
		\includegraphics[width=0.80\textwidth]{images/wide_flour_spher_All.jpg}
	\caption{Images of a $\unit[200]{nm}$ bead $\unit[67]{\upmu m}$ below the 
cover slip in a water/glycerol mixture with n = 1.42.  (a) Uncorrected image 
of in-focus plane. (b) Corrected image of in-focus plane. Images (c) and (d) 
are the same as (a) and (b), respectively, but on a logarithmic scale for 
better visualization. (e) and (f) are line profiles of the intensity through 
the center of the bead along the lateral and the longitudinal axis, 
respectively. The dashed line is from the uncorrected image and the solid 
line is from the corrected image. (g) and (h) are simulations of the PSF. 
Images based on \cite{wide_AOM_FM_spehrical_correction}.}
	\label{fig:wide_flour_spher_All} 
\end{figure}


The first is the effect of uncorrected aberrations from the sample and the 
optical path, which decrease the maximum intensity at the cover slip, but in 
a way that does not add linearly to the depth aberration. Thus only a 
fraction of the dispersed photons can be restored to the central peak. In 
closed loop AO systems, system aberrations are automatically compensated at 
each position (Wright et al., 2007), but in an open-loop system this is not 
possible. The second factor is the inability of the mirror to precisely 
conform to the shape given by Eq. (1). The residual error of the mirror shape 
increases with depth (see Fig. 3c) so that as the imaging plane goes deeper 
and the possibility for improvement becomes greater, the improvement in peak 
intensities decreases

Lastly, the ultimate goal of applying adaptive optics in microscopy is to 
correct all aberrations including those introduced by the refractive index 
variations of the sample itself.
\cite{wide_AOM_FM_spehrical_correction}