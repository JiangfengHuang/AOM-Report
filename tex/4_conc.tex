%%%%%%%%%%%%%%%%%%%%%%%%%%%%%%%%%%%%%%%%%%%%%%%%%%%%%%%%%%%%%%%%%%%%%%%%%%%%%%%
\section{Conclusion \& Future Prospects}
\label{sec:Future}
%%%%%%%%%%%%%%%%%%%%%%%%%%%%%%%%%%%%%%%%%%%%%%%%%%%%%%%%%%%%%%%%%%%%%%%%%%%%%%%

While adaptive optics has been applied in many fields for more than fifteen years, in biological imaging it is still a relatively new system. Since new AO techniques for modern microscopes are just being developed, it will take more time before AO can become a standard component of laboratory microscopes.  

It is important to remark that AO applied to microscopy does not overcome the diffraction limit but rather helps to restore a diffraction limited imaging case. It therefore usually yields in an improvement in both axial and lateral resolution as well as an increase in signal intensity. In other words, AO extends the capabilities of high-resolution or superesolution techniques, especially extending the ability to image deeper into tissue. The advantages of correcting aberrations always depend on the specimen under study and the microscopy technique used. One of the most critical issues in AO applied to microscopy is the accuracy in the sensing process, that is, how we obtain the aberration information from the sample. It seems to be that indirect sensing is the first option in almost all the experiments for AO microscopes. This is because it is easier to implement in microscopy since it only requires a deformable mirror. Indirect sensing is however usually slower than direct sensing. 

An important drawback in most AO methods for microscopy is the relatively slow wavefront detection. As long as aberrations don't change quickly or when just considering a small region of a sample, this is not a problem. However, when imaging larger parts of a sample or when trying to image fast processes in live samples, AO techniques are ofter to slow. Thus, it is important to further optimize current system or to develop completely new, faster correction and sensing devices~(using multiple correctors being a possible solution). This would extend the ability to correct aberrations in real time and with less exposure of the specimens during the measurement. Furthermore, AO methods have not been implement in techniques such as STORM and PALM yet. First publications in these fields are only a matter of time and are probably due to the young age of the methods, as AO techniques are currently being developed~\cite{future_AOM_PALM_1}. \newline

\noindent It is difficult to give specific advice regarding the choice of the correct adaptive optics system. If measurement speed and photobleaching are a concern, direct sensing is should be considered, with the possible downside of a higher complexity and the need for a guide star. 
If a maximal signal intensity is more important and photobleaching is not a problem and if space, hardware and budget are limited, indirect sensing might be the better choice. It offers a cheap and easy AO implementation if time is not a concern or if suitable optimization algorithms can be the developed. In the the end, the right choice has to be  determined based on the microscopy technique and system constrains. Despite all this, further improvements in the sensing and correction process will certainly further extend the already wide range AO applications in modern microscopes in the future. 


%Further advances in AO will extend the capabilities of high-resolution 
%microscopes to reveal functional and structural information from deep within 
%biological tissue. Currently, optimum performance is often limited to thin 
%regions near to the coverslip, sufficient for imaging individual cells, but 
%of rather limited practicality for tissue imaging. AO promises to help move 
%microscopy into a new regime in which biological studies that were previously 
%confined to cell cultures can be performed in thick tissue and even in live 
%specimens.

%abberations that are important for imaging system differ a lot, some need 
%correction of both illu and imaging, some use direct, other indirect methods, 
%some only need correction of some abberation modes, others need as much as 
%possible...important to choose correct system for given problem and figure it 
%our in detail before ordering something
%- do your reserach before you apply a AO system, what is important
%-> signal -> use indirect sensing
%-> speed -> direct sensing
%-> photobleaching a problem? -> direct
%-> photobleaching no problem and fast acq. time -> indirect and random
%-> hardware / space / money limited -> indirect
%Maybe think of guide to choose correct AO system...

%- wavefront sensorless approach was taken to imaging a green fluorescent protein (GFP) labelled transgenic zebrafish
%- first to correct large aberrations, contrast of the image was used as the metric then after several correction iterations, the high spatial frequency content was used
%- aberrations come from varying depths
%- AO on the imaging path can be used to correct for focus and higher order aberrations on the imaging path as well as beam displacement on both the illumination and imaging paths
%- not giving real results yet
%- use better algorithms for CARS, make it a LOT faster


