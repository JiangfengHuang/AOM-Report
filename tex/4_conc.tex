%%%%%%%%%%%%%%%%%%%%%%%%%%%%%%%%%%%%%%%%%%%%%%%%%%%%%%%%%%%%%%%%%%%%%%%%%%%%%%%
\section{Future Prospects}
\label{sec:Future}
%%%%%%%%%%%%%%%%%%%%%%%%%%%%%%%%%%%%%%%%%%%%%%%%%%%%%%%%%%%%%%%%%%%%%%%%%%%%%%%

More work must be done before AO can become a regular component of laboratory 
microscopes. Most AO microscopes are too complex to set up, and their 
application can be limited by the robustness of operation. The development of 
automated alignment and calibration procedures would enable the turnkey 
operation needed to make these systems more practical.

The effectiveness of AO microscopy is mostly compromised by aberration 
measurement, rather than by currently available correction devices. More 
sophisticated wavefront sensors or sensorless optimization schemes will 
extend the microscope’s ability to cope with large and more complex 
aberrations. An obvious goal is to develop “realtime” aberration sensing to 
increase the speed of correction. Coupled to this is the desire to reduce the 
exposure of specimens during the measurement process—an essential step when 
using microscopes for live imaging.

Aberrations can change significantly across a single field of view because 
the refractive index of the specimen varies throughout its volume. So far, 
the methods used in adaptive microscopes have provided only a fixed 
aberration correction for each image. This is sufficient if the imaged region 
is small enough that aberrations do not vary significantly across the field.

One way to overcome this limitation would be to apply multiconjugate AO to 
microscopes. This method has been applied in astronomy using multiple 
deformable mirrors to compensate for multiple aberrating layers in the 
atmosphere. A similar approach in microscopy would compensate for the 3-D 
refractive index distribution, although the optical system would become 
considerably more complex.

Further advances in AO will extend the capabilities of high-resolution 
microscopes to reveal functional and structural information from deep within 
biological tissue. Currently, optimum performance is often limited to thin 
regions near to the coverslip, sufficient for imaging individual cells, but 
of rather limited practicality for tissue imaging. AO promises to help move 
microscopy into a new regime in which biological studies that were previously 
confined to cell cultures can be performed in thick tissue and even in live 
specimens.

\textbf{Add short part about STORM/PALM which are not using adaptive optics yet...
} \cite{future_AOM_PALM_1}
%The Conclusion section presents the outcome of the work by interpreting the findings at a higher level of abstraction than the Discussion and by relating these findings to the motivation stated in the Introduction.
%%%%%%%%%%%%%%%%%%%%%%%%%%%%%%%%%%%%%%%%%%%%%%%%%%%%%%%%%%%%%%%%%%%%%%%%%%%%%%%
\section{Conclusion}
\label{sec:Conclusion}
%%%%%%%%%%%%%%%%%%%%%%%%%%%%%%%%%%%%%%%%%%%%%%%%%%%%%%%%%%%%%%%%%%%%%%%%%%%%%%%

abberations that are important for imaging system differ a lot, some need 
correction of both illu and imaging, some use direct, other indirect methods, 
some only need correction of some abberation modes, others need as much as 
possible...important to choose correct system for given problem and figure it 
our in detail before ordering something

AO is applicable to almost all microscopy techniques...

%TODO
Is image based aberration detection and correction as performed in \cite{wide_AOM_loew_freq} and \cite{wide_AOM_structured_illu} done in point scan techniques?


\clearpage