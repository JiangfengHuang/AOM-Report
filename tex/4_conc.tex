%%%%%%%%%%%%%%%%%%%%%%%%%%%%%%%%%%%%%%%%%%%%%%%%%%%%%%%%%%%%%%%%%%%%%%%%%%%%%%%
\section{Conclusion \& Future Prospects}
\label{sec:Future}
%%%%%%%%%%%%%%%%%%%%%%%%%%%%%%%%%%%%%%%%%%%%%%%%%%%%%%%%%%%%%%%%%%%%%%%%%%%%%%%

Adaptive Optics have been applied in many fields for more than fifteen years, however, in biological imaging it is relatively new to use AO systems. It still needs more research before AO can become a regular component of laboratory microscopes, the particular characteristics of microscopes make AO specially difficult to implement it.  

The first think it is important to remark in AO applied to microscopy is that correcting aberrations does not give us high resolution or superesolution by itself. It gives us an improvement in axial and lateral resolution as well as an increase in signal intensity, in other words, AO extend the capabilities of high-resolution or superesolution techniques. Moreover, the advantages of correcting aberrations will always depend on the specimen under study and the microscopy technique used. One of the most critical issues in AO applied to microscopy is the accuracy in the sensing process, that is, how we obtain the aberration information from the sample. It seems to be that indirect sensing is the first option in almost all the experiments for AO microscopes, that is because it is more suitable and easy to implement in microscopy since only requires a deformable mirror, although it is less faster than direct sensing. Regarding the choice of a direct or indirect methods, we can say in general terms that, if we care more about the speed of measurement and photobleaching is a problem, direct sensing is likely to be the right choice. But, if care more about the signal intensity and photobleaching is not a problem in our experiment, indirect sensing might be the right choice. Although at the end, the right choice will be determined by the microscopy technique as well. Despite all this, further improvements in the sensing process will extend the microscope’s ability to image samples with large and more complex aberrations.

Another important aspect in AO for microscopy is that aberrations don't change quickly when we just consider a small region of a biological sample. In this sense, the time response of the current corrector devices is enough. However, as long as we want to scan different parts of the sample, aberrations can change significantly because the refractive index of the specimen varies throughout its volume. Thus, we need to implement in a microscope a faster corrector device or multiple corrector devices in the same AO system. This last solution still needs to be further investigated in microscopy. Also, to our knowledge, AO applications in techniques such as STORM/PALM need still to be experimented. Finally, another future prospect that is essential for live imaging is that further improvements in both, the speed of the sensing process and the time response of the corrector devices, will extend the ability to correct aberrations in real time and with less exposure of the specimens during the measurement. 



%Further advances in AO will extend the capabilities of high-resolution 
%microscopes to reveal functional and structural information from deep within 
%biological tissue. Currently, optimum performance is often limited to thin 
%regions near to the coverslip, sufficient for imaging individual cells, but 
%of rather limited practicality for tissue imaging. AO promises to help move 
%microscopy into a new regime in which biological studies that were previously 
%confined to cell cultures can be performed in thick tissue and even in live 
%specimens.

%abberations that are important for imaging system differ a lot, some need 
%correction of both illu and imaging, some use direct, other indirect methods, 
%some only need correction of some abberation modes, others need as much as 
%possible...important to choose correct system for given problem and figure it 
%our in detail before ordering something
%- do your reserach before you apply a AO system, what is important
%-> signal -> use indirect sensing
%-> speed -> direct sensing
%-> photobleaching a problem? -> direct
%-> photobleaching no problem and fast acq. time -> indirect and random
%-> hardware / space / money limited -> indirect
%Maybe think of guide to choose correct AO system...

%- wavefront sensorless approach was taken to imaging a green fluorescent protein (GFP) labelled transgenic zebrafish
%- first to correct large aberrations, contrast of the image was used as the metric then after several correction iterations, the high spatial frequency content was used
%- aberrations come from varying depths
%- AO on the imaging path can be used to correct for focus and higher order aberrations on the imaging path as well as beam displacement on both the illumination and imaging paths
%- not giving real results yet
%- use better algorithms for CARS, make it a LOT faster


