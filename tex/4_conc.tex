%%%%%%%%%%%%%%%%%%%%%%%%%%%%%%%%%%%%%%%%%%%%%%%%%%%%%%%%%%%%%%%%%%%%%%%%%%%%%%%
\section{Conclusion \& Future Prospects}
\label{sec:Future}
%%%%%%%%%%%%%%%%%%%%%%%%%%%%%%%%%%%%%%%%%%%%%%%%%%%%%%%%%%%%%%%%%%%%%%%%%%%%%%%

While adaptive optics has been applied in many fields for more than fifteen years, in biological imaging it is still a relatively new system. Since new AO techniques for modern microscopes are just being developed, it will take more time before AO can become a standard component of laboratory microscopes.  

It is important to remark that AO applied to microscopy does not overcome the diffraction limit but rather helps to restore a diffraction limited imaging case. It therefore usually yields in an improvement in both axial and lateral resolution as well as an increase in signal intensity. In other words, AO extends the capabilities of high-resolution or superesolution techniques, especially extending the ability to image deeper into tissue. The advantages of correcting aberrations always depend on the specimen under study and the microscopy technique used. One of the most critical issues in AO applied to microscopy is the accuracy in the sensing process, that is, how we obtain the aberration information from the sample. It seems to be that indirect sensing is the first option in almost all the experiments for AO microscopes. This is because it is easier to implement in microscopy since it only requires a deformable mirror and no point-like emitter is needed. Indirect sensing is however usually slower than direct sensing. 

An important drawback in most AO methods for microscopy is the relatively slow wavefront detection. As long as aberrations don't change quickly or when just considering a small region of a sample, this is not a problem. However, when imaging larger parts of a sample or when trying to image fast processes in live samples, AO techniques are often slow. Thus, it is important to further optimize the current systems or to develop a completely new, faster correction devices~(using multiple correctors being a possible solution). This would extend the ability to correct aberrations in real time and with less exposure of the specimens during the measurement. The slow sensing is also an important drawback for aberration correction in an industrial microscope used in hospitals or biological labs. Thorlabs offers a ready-made adaptive optics system for TPFM~\cite{future_thorlabs} but it also aims at research labs and is not suitable for the end user. Big microscope manufacturers like Zeiss and Leica have not yet presented a single commercial microscope using adaptive optics, even though Zeiss owns a patent on the technology~\cite{future_zeiss_patent}. Furthermore, AO methods have not been implemented in techniques such as STORM and PALM yet. First publications in these fields are only a matter of time and are probably due to the young age of the methods, as AO techniques are currently being developed~\cite{future_AOM_PALM_1}. \newline

\noindent It is difficult to give specific advice regarding the choice of the correct adaptive optics system. If measurement speed and photobleaching are a concern, direct sensing should be considered, with the possible downside of a higher complexity and the need for a guide star. If a maximal signal intensity is more important, photobleaching is not a problem and if space, hardware and budget are limited, indirect sensing might be the better choice. It offers a cheap and easy AO implementation if time is not a concern or if suitable optimization algorithms can be developed. In the end, the right choice has to be determined based on the microscopy technique and system constrains. Despite all this, further improvements in the sensing and correction process will certainly further extend the already wide range AO applications in modern microscopes in the future. 