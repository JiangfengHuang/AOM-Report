\section*{Abstract}

To achieve optimal, diffraction limited images with the best possible resolution has always been a challenging topic and an important goal in science and in biophotonics in particular. Although many new techniques to acquire superesolution images were developed in the last decade, unwanted aberrations often still compromise the resolution and brightness of the images taken with these techniques. Adaptive Optics~(AO), a technique originated from astronomy, is a possible solution to compensate for aberrations in biomedical imaging. 

This report presents a review of the numerous applications of AO applied to modern microscopy methods. We review the basic principles of AO, describe how the aberration information is obtained~(via direct and indirect sensing) and how AO is applied in the main microscopy techniques~(widefield and point scanning techniques). We show that AO methods have been applied in almost all modern microscopy techniques. However, the increase in resolution and signal intensity depends strongly on both the AO system and the microscopy technique, thus, great care has to be taken when trying to implement the optimal AO technique.