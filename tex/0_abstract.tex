\section*{Abstract}

Images with better resolution have always been a challenging topic and an important goal in many fields of science. Specially, in biophotonics. Although new techniques of superesolution are appearing and being developed, in any optical system there are always aberrations that often compromise the resolution and brightness of images. Thus, to overcome this problem it is used Adaptive Optics~(AO), which is a relatively new field in biophotonics imaging. In this paper we have reviewed the basic principles of AO, focusing our interest in how we obtain the aberration information~(i.e., direct and indirect sensing) and particularly, how AO is applied in the main microscopy techniques~(i.e., widefield and point scanning techniques). We have shown that in almost all techniques AO have been used. However, the increase in resolution depend on many factors such as the specimen under study and the microscopy technique used. 
  
%TODO - find reference style that is small but still shows url/doi
