%The Introduction section clarifies the motivation for the work presented and 
%prepares readers for the structure of the paper.
% context:  orient those readers who are less familiar with your topic and to 
%establish the importance of your work
% need: state the need for your work, as an opposition between what the 
%scientific community currently has and what it wants.
% task: indicate what you have done in an effort to address the need (this is 
%the task)
% object: preview the remainder of the paper to mentally prepare readers for 
%its structure, in the object of the document
%%%%%%%%%%%%%%%%%%%%%%%%%%%%%%%%%%%%%%%%%%%%%%%%%%%%%%%%%%%%%%%%%%%%%%%%%%%%%%%
\section{Introduction}
\label{sec:Introduction}
%%%%%%%%%%%%%%%%%%%%%%%%%%%%%%%%%%%%%%%%%%%%%%%%%%%%%%%%%%%%%%%%%%%%%%%%%%%%%%%

It is well known that optical aberrations degrade the resolution and brightness of images. That means a reduction in both lateral and axial resolution and a fall in signal intensity. Aberrations, in general terms, can be defined as the wavefront distortions with respect to ideal spheres, these distortions are due to imperfections in any part of the optical system. In microscopy, aberrations may arise from the microscope itself or the specimen under study \cite{AOM_basic_ref}. Therefore, aberrations always limit in some way the final image quality and vary from one specimen to another, so they cannot be corrected by a fixed optical design. Dynamic correction is necessary. 
 
That is the reason why scientists have been trying to overcome this problem for many years. The best approach to it is what nowadays is called Adaptive Optics (AO).
The first proposal of the use of AO technology was suggested in the year 1953 in the context of astronomical optics for the compensation of the aberrating effects of the atmosphere \cite{Babcock1953}.

The main idea of the AO is the modulation of an incoming wavefront in such a way that we can record an image without aberrations. It is based upon the principle of phase conjugation: the correction element introduces an equal but opposite phase aberration to that present in the optical system. In order to do that, we need to be able to measure these distortions reliably. The most direct way is to use a wavefront sensor, the Shack-Hartmann \cite{Principles_HS, History_HS} or Curvature Sensors \cite{Curvature_Sensor_1st_Paper} are examples of it. Also, interferometric techniques have been used to measure aberrations \cite{Interferometric_methods}. Nevertheless, there are indirect methods in which aberrations are estimated using and algorithm and do not employ a wavefront sensor, they are called sensorless techniques \cite{WF_sensorless}. 
In addition to the wavefront sensing, we need the adaptive element to modulate the aberrations before the light reaches the imaging detector, this is usually a deformable mirror or a liquid crystal spatial light modulator (LC-SLM). Finally, we need a control system that processes the aberrations information and uses it to monitor the adaptive correction element.

Although Adaptive Optics systems have been introduced in applications such as astronomy, laser beam shaping, optical communications, data storage and ophthalmology \cite{AOM_biomedical}, when it is applied to microscopy is not always trivial and it requires a different approach than in the other fields. One particularly difficult thing in AO microscopy is how the aberration information is obtained in each of the different microscopy techniques.

The optical microscope techniques can be divided in two main groups: the widefield techniques and the point scanning techniques. Some examples of the first group are the conventional transmission microscopy, the structured illumination microscopy and the fluorescence microscopy. Some examples within the second group are the confocal microscopy or the non-linear microscopy such as Two-Photon Excitation Fluorescence (TPEF), Second Harmonic Generation (SHG), Third Harmonic Generation (THG), Coherent anti-Stokes Raman (CARS) or Stimulated Emission Depletion (STED).  

In this report, in order to explain how adaptive optics and microscopy link together, we will start explaining the basis of adaptive optics. That means a brief review of the concept of aberrations and how they are most commonly characterized. After that, we will explain in more detail the main methods for wavefront sensing, the main aberration corrector devices and some control strategies. Finally, we will show some applications of the AO in different widefield and point scanning microscopy techniques. The last part will be a short explanation of future prospects and conclusions.       
