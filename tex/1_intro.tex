%The Introduction section clarifies the motivation for the work presented and 
%prepares readers for the structure of the paper.
% context:  orient those readers who are less familiar with your topic and to 
%establish the importance of your work
% need: state the need for your work, as an opposition between what the 
%scientific community currently has and what it wants.
% task: indicate what you have done in an effort to address the need (this is 
%the task)
% object: preview the remainder of the paper to mentally prepare readers for 
%its structure, in the object of the document
%%%%%%%%%%%%%%%%%%%%%%%%%%%%%%%%%%%%%%%%%%%%%%%%%%%%%%%%%%%%%%%%%%%%%%%%%%%%%%%
\section{Introduction}
\label{sec:Introduction}
%%%%%%%%%%%%%%%%%%%%%%%%%%%%%%%%%%%%%%%%%%%%%%%%%%%%%%%%%%%%%%%%%%%%%%%%%%%%%%%

It is well known that optical aberrations degrade the resolution and brightness of images. This results in a reduction of both lateral and axial resolution and a decrease in signal intensity. In general aberrations can be defined as the wavefront distortions with respect to an ideal sphere. These distortions can be caused by imperfections and inhomogeneities in any part of the optical system. In microscopy, aberrations may arise from the microscope itself or the specimen under study~\cite{AOM_basic_ref}. Aberrations always limit the final image quality and can vary from one specimen to another. In this case they can not be corrected by an optimized optical design which makes a dynamic correction necessary. 
 
It is of little surprise that scientists have been trying to overcome this problem for many years, an effort that resulted in what nowadays is called Adaptive Optics~(AO). The first proposal of the use of AO technology was suggested in the year 1953 in the context of astronomical optics for the compensation of the aberrating effects of the atmosphere~\cite{Babcock1953}. The main idea of AO is the modulation of an incoming wavefront in such a way that we can record an image without aberrations. It is based upon the principle of phase conjugation: the correction element introduces an equal but opposite phase aberration to that present in the optical system. In order to do that, we need to be able to measure these distortions reliably. The most direct way is to use a wavefront sensor, such as the Shack-Hartmann~\cite{Principles_HS, History_HS}. Also, interferometric techniques have been used to measure aberrations~\cite{Interferometric_methods}. Furthermore, there are indirect or sensorless methods in which aberrations are determined using and optimization algorithm and do not employ a direct  wavefront sensing~\cite{WF_sensorless}. A control system then processes the aberration information and uses it to control an adaptive correction element. The adaptive element is needed to modulate and correct the incoming wavefronts before the light reaches the imaging detector. This task is usually performed by a deformable mirror or a liquid crystal spatial light modulator (LC-SLM). 

Although Adaptive Optics systems have been successfully introduced in applications such as astronomy, laser beam shaping, optical communications, data storage and ophthalmology~\cite{AOM_biomedical}, it is not trivially applied to microscopy. One particularly difficult problem in AO microscopy is how the aberration information is obtained in each of the different microscopy techniques, since direct sensing is not usually easily implemented as it is the case of astronomy.

Optical microscope techniques can be divided in two main groups: the widefield techniques and the point scanning techniques. Examples of the first group are the conventional transmission microscopy, the structured illumination microscopy and the fluorescence microscopy. Some point scanning techniques are the confocal microscopy, Stimulated Emission Depletion~(STED) or the non-linear microscopy such as Two-Photon Excitation Fluorescence~(TPEF), Second Harmonic Generation~(SHG) microscopy, Third Harmonic Generation~(THG) microscopy and Coherent anti-Stokes Raman~(CARS). \newline

In order to explain how adaptive optics and microscopy are linked together, we will start explaining the basis of adaptive optics. This includes a brief review of the concept of aberrations and how they are most commonly characterized. After that, we will explain in more detail the main methods for wavefront sensing, the main aberration corrector devices. Furthermore, we will show some applications of the AO in different widefield and point scanning microscopy techniques. The last part will be a short explanation of future prospects and conclusions.       
