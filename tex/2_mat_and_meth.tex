%The Materials and Methods section provides sufficient detail for other 
%scientists to reproduce the experiments presented in the paper. In some 
%journals, this information is placed in an appendix, because it is not what 
%most readers want to know first.

% explicit preview would be phrased much like the object of the document: 
%"This section first . . . , then . . . , and finally . . . "
% Do not make readers guess: Make sure the paragraph's first sentence gives 
%them a clear idea of what the entire paragraph is about.
%%%%%%%%%%%%%%%%%%%%%%%%%%%%%%%%%%%%%%%%%%%%%%%%%%%%%%%%%%%%%%%%%%%%%%%%%%%%%%%
\section{Adaptive Optics Methods applied in Microscopy}
\label{sec:ExperimentDiscussion}


\cite{adaptive_optics_bio_microscope}


%%%%%%%%%%%%%%%%%%%%%%%%%%%%%%%%%%%%%%%%%%%%%%%%%%%%%%%%%%%%%%%%%%%%%%%%%%%%%%%
\subsection{Widefield Microscopy}
\label{sec:WidefieldMicroscopy}

In conventional microscopes, widefield illumination is provided using either 
transmission optics or, in the case of re ection or uorescence modes, via the 
objective lens in an epi configuration. In either case, the image quality 
depends only on the optics of the detection path and is independent of the 
fidelity of the illumination path.
Aberration correction is therefore only necessary in the detection path and a 
single pass adaptive optics system will suffice.

%-------------------------------------------------------------------------------
\subsection{Point Scanning Microscopes}
\label{sec:PointScanningMicroscopes}

Scanning optical microscopes are widely used for high resolution imaging, 
mainly because certain implementations provide three-dimensional resolution 
with optical sectioning and are thus particularly useful for imaging the 
volume structures of biological specimensz. In these microscopes, 
illumination is provided by a laser that is focused by an objective lens into 
the specimen. The light emitted from the specimen is collected, usually 
through the same objective lens, and its intensity is measured by a single 
photodetector. The focal spot is scanned through the specimen in a raster 
pattern and the image is acquired in a point-by-point fashion. The resulting 
data are stored and rendered as images in a computer.

Several other point scanning microscope modalities have been introduced, 
including two-photon excitation uorescence (TPEF) microscopy, second harmonic 
generation (SHG) and third harmonic generation (THG) microscopy, and coherent 
anti-Stokes Raman (CARS) microscopy.

%-------------------------------------------------------------------------------
\subsubsection{Confocal Microscopes}
\label{sec:ConfocalMicroscopes}

The most common example of this type is the confocal microscope, which can be 
operated in reflection or fluorescence mode. Three-dimensional resolution is 
achieved by the placement of a pinhole in front of the photodetector. In a 
reflection mode confocal microscope, the illumination is scattered by objects 
not only in the focal region, but throughout the focusing cone. In 
fluorescence mode, emission is generated in the focus but also in out-of-
focus regions. The pinhole ensures that mainly light from the focal region 
falls upon the detector and light from out-of-focus planes is obscured. It is 
critical in the confocal microscope that both the illumination and detection 
paths are diffraction limited. This ensures that i) the illuminating focal 
spot is as small as possible, and ii) that the focus is perfectly imaged on 
to the detector pinhole. Therefore, in an adaptive confocal microscope, 
aberration correction must be included in both paths. This dual pass adaptive 
system can usually be implemented using a single deformable mirror, if the 
path length aberrations are the same for both the illumination and the 
emission light. This is the case if there is no significant dispersion in the 
specimen or chromatic aberration in the optics.

A pinhole is not required to obtain three-dimensional resolution, so most 
TPEF microscopes use large area detectors to maximise signal collection. 
Although they rely upon other physical processes, non-linear imaging 
modalities such as SHG, THG and CARS exhibit similar resolution properties. 
When using large area detectors, the fidelity of imaging in the detection 
path is unimportant so the effects of any aberrations in this path are 
negated. It follows that single pass adaptive optics is appropriate for these 
microscopes as aberration correction need only be implemented in the 
illumination path.

Adaptive optics systems have been successfully combined with several point-
scanning microscope systems including confocal,13 TPEF,6, 14, 15 harmonic 
generation,16, 17 CARS.18 Example images of aberration correction in an 
adaptive THG microscope are shown in Fig. 10.

\cite{book_confocal}

\cite{confocal_microscope}


%-------------------------------------------------------------------------------
\subsubsection{two-photon excitation}
\label{sec:twoPhotonExcitation}

\cite{TPFM_gated_wavefront}
\cite{TPFM_image_based}
\cite{TPFM_pratical}

%-------------------------------------------------------------------------------
\subsubsection{Harmonic Generation}
\label{sec:HarmonicGeneration}

\cite{HG_embryos}
\cite{HG_dynamic}


%-------------------------------------------------------------------------------
\subsubsection{CARS}
\label{sec:CARS}

\cite{CARS}



